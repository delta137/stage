\documentclass[handout]{beamer}
%
%\mode<presentation>
%{
  \usetheme{Copenhagen} %type de présentation
  %\usecolortheme{blue}% couleur

  %\setbeamercovered{transparent} %laisse le texte à paraître en gris
%}
\usepackage[T1]{fontenc}
\usepackage[utf8]{inputenc}
\usepackage{lmodern}
\usepackage[frenchb]{babel}
\usepackage{amsmath}
\usepackage{amsfonts}
\usepackage{amssymb}
\usepackage{cancel}

\frenchspacing

\allowdisplaybreaks %?
\let\Tiny=\tiny %pour enlever message erreur: Font shape `OT1/cmss/m/n? in size <4> not available
                %(Font) size <5> substituted on input line...

%\beamerdefaultoverlayspecification{<+->}


%-----------------------------------------------------------------------------------------------------------------
\title{Solitons et cosmologie\\ ou symétrie et beauté intersidérale}
\subtitle{Conférence du vendredi}
\author{Éric Dupuis}
\institute{Université de Montréal, département de physique des particules}
\date{XX-XX-2014}


%-----------------------------------------------------------------------------------------------------------------

\begin{document}

%titre
\begin{frame}
\titlepage
\end{frame}
%

\section*{}
\begin{frame}
\tableofcontents
\end{frame}



\section{Cosmologie}

\begin{frame}
\frametitle{Cosmologie 101}
\begin{enumerate}
\item Cosmologie: Étude de la structure/origine/évolution de l'univers
\item Constantes de couplage, paramètres libres
\item Univers primordial et théorie d'unification (particules)
\item Atteinte du vide: brisure spontanée de symétrie
\end{enumerate}
\end{frame}

\subsection{Symétrie des groupes de Lie}
\begin{frame}
\frametitle{Symétries en physique des particules}
Modèle standard permet les associations suivantes:
\begin{block}{Groupes de Lie pour les forces}
\begin{enumerate}
\item nucléaire faible: SU(2)
\item nucléaire forte: SU(3)
\item électromagnétique: U(1)
\end{enumerate}
\end{block}
\begin{block}{Brisure spontanée de symétrie}
Les lois de la nature peuvent posséder des symétries sans que l'état de vide (fondamental) le soit nécessairement
\begin{enumerate}
\item Boson: Goldstone, Higgs et les jauges
\item Exemple: Glashow, Salam et Weinberg: $SU(2)xU(1)$ 
\end{enumerate}
\end{block}
\end{frame}

\begin{frame}
\frametitle{Un exemple pour les copains de Mat Con}
Exemple: aimantation dans un matériau ferromagnétique \\
Hamiltonien d'intéraction: 
\begin{columns}
    \begin{column}{.5\linewidth}
   \begin{figure}[0.3\textwidth]
   % \includegraphics[scale=0.25]{aimants.jpg}
    \end{figure}
    \end{column}
    \begin{column}{.5\linewidth}
    \begin{figure}[0.3\textwidth]
   %\includegraphics[scale=0.25]{chapeau_mex.png}
    \end{figure}
    \end{column}
  \end{columns}
Et bien plus encore: vortex pour expliquer les supra type I et II
\end{frame}

\subsection{Symétrie et cosmologie}
\begin{columns}
    \begin{column}{.5\linewidth}
   \begin{figure}[0.3\textwidth]
  %  \includegraphics[scale=0.25]{pot_uni.jpg}
    \end{figure}
    \end{column}
    \begin{column}{.5\linewidth}
    \begin{enumerate}
    \item max: univers primordial (densité d'énergie)
    \item min: notre état de l'univers?
    \item transition, et apparition d'un vrai vide
    \end{enumerate}
    \end{column}
  \end{columns}
  
\begin{columns}
\begin{column}{.5\linewidth}
De retour aux aimants:
    \begin{enumerate}
    \item L'atteinte d'une température critique (Curie)
    \item Différentes configurations de vide prises
    \item Rencontre: murs de domaine
    \end{enumerate}
    \end{column}
	\begin{column}{.5\linewidth}
    \begin{figure}[0.3\textwidth]
   % \includegraphics[scale=0.3]{defauts_topo.png}
    \end{figure}
	\end{column}
	
\end{columns}

  
\begin{frame}
Cordes cosmiques: Ligne 1d (quasi) \\
Défauts topologiques: brisure de symétrie cylindrique \\
Même objet que les lignes de vortex, supra \\
 
\end{frame}





%-----------------------------------------------------------------------------------------------------------------
\section{Solitons - Appareillage mathématique }

\subsection{Équation d'ondes et soliton}
%eq d'onde
\begin{frame}
\frametitle{Équation d'ondes}
\begin{itemize}
\item  champ scalaire défini dans $\mathbb{R}^n$: $\phi(\vec{x},t)$\\
\end{itemize}
\begin{block}{équation d'onde}
\begin{equation}
\frac{1}{c^2}\frac{\partial^2 \phi}{\partial t^2} - \nabla^2 \phi = \square \phi = 0 
\end{equation}
\textit{Deux propriétés étudiées dans les solutions $\phi$}
\begin{itemize}
\item Forme et vitesse de l'onde conservées\\
\item Deux ondes retrouvent asymptotiquement leur forme et vitesse\\
\end{itemize}
\end{block} 
\end{frame}

%potentiel de solitons
\begin{frame}
\begin{enumerate}
\item Équation d'onde: V=0
\end{enumerate}

\begin{block}{Potentiels différents - Équations du mouvements modifiées}
\begin{enumerate}

\item terme dispersif: $\square\phi +$ \boldmath $m^2 \phi $ \unboldmath  = 0 (Klein-Gordon)\\
\begin{enumerate}
\item onde plane: $k^2 \rightarrow k^2+m^2$

\end{enumerate}
\item terme non-linéaire: $\phi^3$ \\
\end{enumerate}
\end{block}
En s'éloignant de l'équation d'onde, les deux propriétés peuvent être conservées: ondes solitaires et solitons 

\end{frame}

%photo Russell pour le plaisir
\begin{frame}
Sur sa monture, John Russell poursuit sa destinée qui le mène vers l'onde solitaire
\begin{columns}
    \begin{column}{.5\linewidth}
   \begin{figure}[0.3\textwidth]
    % \includegraphics[scale=0.25]{russell.png}
    \end{figure}
    \end{column}
    \begin{column}{.5\linewidth}
    \begin{figure}[0.3\textwidth]
   %  \includegraphics[scale=0.25]{soliton1b.png}
    \end{figure}
    \end{column}
  \end{columns}
\end{frame}


\begin{frame}
	\begin{enumerate}
	\item Densité d'énergie d'un soliton $\epsilon(x,t)$: localisée dans l'espace
	\item $E[\phi] = \int{dxdt (\mathcal{H}[\phi])} = \int{dxdt [\frac{1}{2} \partial_x \phi (\partial_x \phi)^* +V]}$
	\item Énergie finie:
	\begin{enumerate}
	\item  $\lim\limits_{x \to \pm\infty}\partial_x E =0$
	\item  $\lim\limits_{x \to \pm\infty} \phi[x] = g^{(i)}$	
	\end{enumerate}	
	\end{enumerate} 
\end{frame}


\subsection{Formalisme Lagrangien}
\begin{frame}

\frametitle{Formalisme Lagrangien}

\begin{block}{Quelques notions pratiques}
\begin{enumerate}

\item Notation covariante:
\begin{enumerate}
\item $x_\mu = (x_0,\vec{x})$
\item $x_0 = ct, x_{1,2,3} = x,y,z$
\item indices répétés: $v_a \cdot v_a = \sum_{i=0}^{3}x_i^2$ (produit scalaire)
\end{enumerate}
\item Métrique: $x^\mu = g^{\nu\mu} x_\nu$
\item Minkowski: $\eta^{\nu\mu}$ diag(1,-1,-1,-1) 
\end{enumerate}
\end{block}
\end{frame}

\begin{frame}
\begin{enumerate}
\item \textit{Action}: $S[\phi] = \int{dt (L[\phi])}  =  \int{dx_\mu (\mathcal{L}[\phi])}$
\begin{enumerate}
\item Principe d'Hamilton: $\phi_0$ | action minimisée \\
\item Premier ordre nul pour un minimum d'action \\
\end{enumerate}
\item  $\mathcal{L}[\phi] = \frac{1}{2} \partial_\mu \phi (\partial^\mu \phi)^* -V$
\item \textit{Euler-Lagrange:} $\partial_\mu \left(\frac{\mathcal{L}}{\partial(\partial_\mu\phi)}\right) = \frac{\partial\mathcal{L}}{\partial\phi}$
\end{enumerate}
\begin{align*}
 V=0 \\
 \rightarrow (E-L)  \partial_\mu(\frac{\partial_a\phi (\partial^a\phi)^* }{\partial_\mu \phi}) = 0 \\
 \partial_t(\partial^{t} \phi )^{*} + \partial_x(\partial^{x} \phi )^{*} +... = 0 \\
 \square \phi = 0 \\
\end{align*}
\end{frame}

\subsection{Kink}
\begin{frame}
\frametitle{Kink: cas de figure typique}
\begin{block}{Potentiel d'ordre 4}

 \begin{columns}
    \begin{column}{.5\linewidth}
    \begin{enumerate}
    \item deux minimums absolus
    \item ...
	\end{enumerate}      
    \end{column}
    \begin{column}{.5\linewidth}
    $V(\phi) = \frac{\lambda}{4}(|\phi|^2 -\frac{m^2}{\lambda})^2$
    \begin{figure}
     %\includegraphics[scale=0.5]{boom.jpg}
    \end{figure}
    \end{column}
  \end{columns}

\end{block}
$\mathcal{L} = \frac{1}{2}(\partial_x \phi)^2 - V $ \\
$\rightarrow \phi'' = \lambda \phi^3 - m^2 \phi$ \\
\end{frame}
\begin{frame} \frametitle{Analogie mécanique classique}
\begin{columns}
    \begin{column}{.55\linewidth}
    Champ
    \begin{enumerate}
    \item $\mathcal{H} =  \cancelto{}{\frac{1}{2}  (\partial_t \phi)^2} +  \frac{1}{2}  (\partial_x \phi)^2 +V(\phi) $
    \item $\frac{\partial^2\phi}{\partial_x^2} = \frac{\partial V}{\partial\phi}$
    \item $E_\phi = \int{dx(\frac{1}{2}  (\partial_x \phi)^2 +V(\phi) )}$
    \end{enumerate}
    \end{column}
    \begin{column}{.45\linewidth}
	Particule
    \begin{enumerate}
    \item $L =   \frac{1}{2}  \dot{q}^2 -U(q)$
    \item $\frac{\partial^2\phi}{\partial_x^2} = -\frac{\partial U}{\partial\phi}$
    \item $S_q = \int{dt[\frac{1}{2}  \dot{q}^2 -U(q)] }$
    \end{enumerate}
    \end{column}
  \end{columns}
   \begin{enumerate}
   \item $E_\phi$ finie $\leftrightarrow S_q$ finie $\rightarrow E_q = 0$
   \item On étudie alors: -V
   \end{enumerate}
  
\end{frame}

\begin{frame}
$V(\pm \infty) \rightarrow \pm 1$
\begin{figure}
%\includegraphics[scale=0.5]{pot_inv.jpg}
\end{figure}
Solution privilégiée: D'un max à l'autre
\end{frame}

\begin{frame}
Récapitulons:
\begin{enumerate}
\item travailler en 1+1 dimensions, mais solution statique
\item Potentiel V $\rightarrow$ Équations d'Euler-Lagrange 
\item $\rightarrow$ équation du mouvement: 
\begin{enumerate}
\item $\rightarrow \phi'' = \lambda \phi^3 - m^2 \phi$ (équation statique)
\item solution: kink
\end{enumerate}
\end{enumerate}
\end{frame}


%-----------------------------------------------------------------------------------------------------------------
\section{Symétrie - Modèles de l'univers }
\begin{frame}

\end{frame}






\section{Quotidien en cosmologie théorique des particules}
\begin{frame}
%potentiel avec lequel on travaille
\begin{block}{Potentiel à deux champs $\phi(x,t)$ et $\psi(x,t)$}
\begin{equation*}
V(\phi,\psi)=(\psi^2-\delta_1)(\psi^2-1)^2+\frac{\alpha}{\psi^2+\gamma}[(\phi^2-1)^2 - \frac{\delta_2}{4}(\phi-2)(\phi+1)^2] 
\end{equation*}
\begin{columns}
    \begin{column}{.5\linewidth}
  
\begin{enumerate}
\item 1+1 dimensions (x,t) mais on cherche une solution statique
\item Beaucoup de paramètres: $\alpha, \gamma, \delta_1, \delta_2$
\item Les champs sont couplés
\end{enumerate}  
    \end{column}
    \begin{column}{.5\linewidth}
    \begin{figure}[0.3\textwidth]
    \includegraphics[scale=0.3]{pot.png}
    \end{figure}
   $\delta_2 \rightarrow$ contrôle de la séparation entre minimum \\
    \end{column}
  \end{columns}
\end{block}
\end{frame}

\begin{frame}
Pourquoi bâtir un potentiel comme ça en premier lieu?!?!\\


\begin{columns}
    \begin{column}{.5\linewidth}
   \begin{figure}[0.3\textwidth]
    % \includegraphics[scale=0.25]{potpsi.png}
    \end{figure}
   $\delta_1 \rightarrow$ contrôle du minima central\\
    Potentiel d'ordre 6, CLASSIQUE!\\    
    \end{column}
    \begin{column}{.5\linewidth}
    \begin{figure}[0.3\textwidth]
     %\includegraphics[scale=0.25]{potphi.png}
    \end{figure}
   $\delta_2 \rightarrow$ contrôle de la séparation entre minimum \\
    \end{column}
  \end{columns}
 $\alpha$: importance 2ème terme \\
 $\gamma$: importance couplage \\
\end{frame}

\end{document}