\documentclass[handout]{beamer}
%
%\mode<presentation>
%{
  \usetheme{Copenhagen} %type de présentation
  %\usecolortheme{blue}% couleur

  %\setbeamercovered{transparent} %laisse le texte à paraître en gris
%}
\usepackage[T1]{fontenc}
\usepackage[utf8]{inputenc}
\usepackage{lmodern}
\usepackage[frenchb]{babel}
\usepackage{amsmath}
\usepackage{amsfonts}
\usepackage{amssymb}
\usepackage{cancel}

\usepackage{array}



\frenchspacing

\allowdisplaybreaks %?
\let\Tiny=\tiny %pour enlever message erreur: Font shape `OT1/cmss/m/n? in size <4> not available
                %(Font) size <5> substituted on input line...

%\beamerdefaultoverlayspecification{<+->}


%-----------------------------------------------------------------------------------------------------------------
\title{Du Big Bang à l'apocalypse:\\ }
\subtitle{Symétries et solitons dans la cosmologie}
\author{Éric Dupuis}
\institute{Université de Montréal, département de physique \\
Conférences du vendredi des stagiaires}
\date{27-06-2014}


%-----------------------------------------------------------------------------------------------------------------

\begin{document}

%titre
\begin{frame}
\titlepage
\end{frame}
%

\section*{}
\begin{frame}
\tableofcontents
\end{frame}



\section{Cosmologie}
\subsection{Cosmologie 101}
\begin{frame}
\frametitle{Cosmologie 101}
\begin{enumerate}
\item Cosmologie: Structure/origine/évolution de l'univers
\begin{enumerate}
\item Big Bang
\item Modèles inflationnistes
\item Expansion de l'univers
\end{enumerate}
\item Règles: Relativité générale, mécanique quantique, théorie des champs
\item Configurations: Constantes de couplage, paramètres libres
\end{enumerate}
\end{frame}


\begin{frame}
\begin{block}{Questions fondamentales (style cosmo)}
\begin{enumerate}
\item D'où venons-nous? (État primordial de l'univers)
\item Qui sommes-nous? (État actuel de l'univers)
\item Où allons-nous? (Évolution de l'univers)
\end{enumerate}
\end{block}
%\begin{figure}[0.5\textwidth]
%  \includegraphics[scale=0.25]{evo_pot.jpg}
%   %+ orientés désorientés ou juste tableau?
% \end{figure}
\end{frame}

\begin{frame}\frametitle{État d'origine $\rightarrow$ État actuel}
\begin{columns}[T]
 \begin{column}[T]{.5\linewidth}
 Big Bang:\\ -Matière compressée\\ -Température très élevée\\ -\textbf{État instable}\\[1 cm]
 \textbf{Forme symétrique:} Importance en cosmologie 
 \end{column}
 \begin{column}[T]{.5\linewidth}
 \begin{figure}[0.5\textwidth]
  \includegraphics[scale=0.25]{chapeau_mex.png}
   %+ orientés désorientés ou juste tableau?
 \end{figure}
 \end{column}
\end{columns}
\end{frame}


%\item Univers primordial et théorie d'unification (particules)
%\item Atteinte du vide: brisure spontanée de symétrie
%image?



\subsection{Notions de symétrie}
\begin{frame}\frametitle{La symétrie en physique}
\begin{block}{Définition (approximative)}
Symétrie: Synonyme d'\textbf{invariance}. La symétrie d'un système physique définit un ensemble de  transformations qui laissent certaines de ses propriétés inchangées (Théorème de Noether).
\end{block}

%\textsc{Théorème de Nother:} Pour toute transformation infinitésimale laissant l'action invariante, il existe une quantité conservée.

\begin{exampleblock}{Symétrie de l'univers (postultats de la relativité restreinte}
\begin{enumerate}
\item Homogénéitié: Invariance sous translation $\rightarrow$ $\vec{p}$ 
\item Isotropie: Invariance sous rotation $\rightarrow$ $\vec{L}$
\end{enumerate}
\end{exampleblock}
\textbf{Groupe:} Opérations de base entre ses éléments.\\  Transformations $\leftrightarrow$ Générateurs du groupe.\\
Exemple: U(n), O(n), SU(n), SO(n)
\end{frame}

\begin{frame}\frametitle{Ferroaimant de Heisenberg: Dipôles magnétiques en 2D}

\begin{columns}[T]
 \begin{column}[T]{.5\linewidth}
 \begin{equation*}
H= -J\sum_{i}{\sum_{voisins j}{\vec{S}_i\cdot\vec{S}_j}}
\end{equation*} 

\begin{figure}[0.5\textwidth]
   \includegraphics[scale=0.25]{potpot.png}
 \end{figure}
 \end{column}
 \begin{column}{.5\linewidth}

 Invariance de H sous rotation dans le plan \textbf{SO(2)} \\[0.5 cm]
 
  \begin{block}{Brisure spontanée de symétrie}
Les lois de la nature peuvent posséder des symétries qui ne laissent toutefois pas l'état de vide (fondamental) invariant.
\end{block}
  \end{column}
\end{columns}
\end{frame}
%
%\begin{frame}
%    \begin{figure}[0.3\textwidth]
%   %\includegraphics[scale=0.25]{aimants_desorientes.jpg}
%    \end{figure}
%    Petit homme dans le réseau
%    
%Et bien plus encore: vortex pour expliquer les supra type I et II
% \end{frame} 

\begin{frame}
\begin{columns}[T]
\begin{column}[T]{.5\linewidth}
    \begin{enumerate}
    \item Température critique (Curie)
    \item Vides dégénérés, différant dans l'espace: défauts topologiques (solitons)
   % \item Rencontre: murs de domaine
    %\item $\rightarrow$ murs: liés aux solitons, défauts topologiques
    \end{enumerate}
    \end{column}
	\begin{column}[T]{.5\linewidth}
    \begin{figure}[0.3\textwidth]
    \includegraphics[scale=0.4]{murs.png}
    \end{figure}
 
	\end{column}
	
\end{columns}
\begin{figure}[0.3\textwidth]
    \includegraphics[scale=0.3]{string.png}
    \end{figure}
\end{frame}


\begin{frame}\frametitle{Symétries en physique des particules}
Modèle standard permet les associations suivantes:\\
\begin{block}{Groupes de Lie, transformations continues}
\begin{tabular}{ccc}
 Force & Groupe de symétrie & Bosons - Générateurs \\
  \hline
  Nucléaire faible & SU(2) & $W^{\pm}$, $Z^0$ \\
  Nucléaire forte & SU(3)& g \\
  Électromagnétique & U(1) &  $\gamma$\\
  &(symétries internes)&
 \end{tabular}
\end{block}
\end{frame}


\begin{frame}\frametitle{Brisure de symétrie - Particules}

\textbf{Mécanisme de Higgs:} Bosons de Goldstone et de jauge \\
\begin{figure}[0.3\textwidth]
    \includegraphics[scale=0.4]{higgs.png}
    \end{figure}

\textbf{Symétrie élecrofaible:}  Glashow, Salam et Weinberg\\
  \begin{equation*}
   SU(2)xU(1)
  \end{equation*}
  
 \textbf{Grande unification:}
\begin{equation*}
G \rightarrow H \rightarrow ... \rightarrow SU(3) x U(1)
\end{equation*}
\end{frame}



\subsection{Symétrie et cosmologie}
%\begin{frame}
%
%\begin{columns}[T]
%    \begin{column}[T]{.5\linewidth}
%   \begin{figure}[0.3\textwidth]
%    %\includegraphics[scale=0.25]{pot_uni.jpg}
%    \end{figure}
%    \end{column}
%    \begin{column}[T]{.5\linewidth}
%    \begin{enumerate}
%    \item max: univers primordial (densité d'énergie)
%    \item min: notre état de l'univers?
%    \item transition, et apparition d'un vrai vide
%    \begin{enumerate}
%    \item possibilité de transition par effet tunnel
%    \item taux de désintégration; dégéneresence des faux vides
%    \end{enumerate}
%    \end{enumerate}
%        %figure Marie-Lou transition du potentiel
%    \end{column}
%  \end{columns}
%  \end{frame}
  
  \begin{frame}\frametitle{Fin de 1ère section}
  
  
  
\begin{columns}[T]
    \begin{column}[T]{.5\linewidth}
    Supposition: Potentiel U\\[0.5 cm]
    \textbf{1)} Origine $\rightarrow$ max(U), Big Bang\\[0.5 cm]
    \textbf{2)} État actuel \\ $\rightarrow$ vide métastable? \\ $\rightarrow$ symétrie brisée?\\[0.5 cm]
	\textbf{3)} Évolution future
	\\ $\rightarrow$ Vers un vrai vide?
	
    \end{column}
    \begin{column}[T]{.5\linewidth}
    \begin{figure}[0.3\textwidth]
    \includegraphics[scale=0.2]{evo_pot.jpg}
    \end{figure}
    
    \end{column}
  \end{columns} 
    \end{frame} 

%    Symétries brisées spontanément: structure non triviale des vides
%    \textbf{Le taux de désintégration de l'univers peut être affecté par la présence de défauts topologiques. Est-ce que l'univers a connu une transition de phase avec brisure de symmétrie? Selon le modèle standard, possiblement. Alors, on attend des cordes cosmiques, monopôles magnétiques. Surtout, ces défauts topologiques pourrait affecter l'évolution actuelle de notre univers. C'est l'objet de notre étude.}
%    



%-----------------------------------------------------------------------------------------------------------------
\section{Solitons - Appareillage mathématique }

\begin{frame}\frametitle{Solitons}
\textbf{Brisure de symétrie:} \\$\rightarrow$
Défauts topologiques: signature des solitons\\[1 cm]

\textbf{Intérêt cosmologique:} 
\\$\rightarrow$ Défaut topologique: configurations stables de matières formées lors de la transition de phase 
\\$\rightarrow$ influence sur le taux de transition vers un vrai vide
      
\end{frame}

\subsection{Équation d'ondes et soliton}
%eq d'onde
\begin{frame}
\frametitle{Équation d'ondes}
Champ scalaire défini dans $\mathbb{R}^n$: $\phi(\vec{x},t)$\\
\begin{block}{Équation d'onde}
\begin{equation}
\frac{1}{c^2}\frac{\partial^2 \phi}{\partial t^2} - \nabla^2 \phi = \square \phi = 0 
\end{equation}
\textit{Deux propriétés étudiées dans les solutions $\phi$}
\begin{itemize}
\item Forme et vitesse de l'onde conservées\\
\item Deux ondes retrouvent asymptotiquement leur forme et vitesse\\
\end{itemize}
\end{block} 
\end{frame}

%potentiel de solitons
\begin{frame}

Équation d'onde: V=0


\begin{block}{Potentiels différents - Équations du mouvements modifiées}
\begin{enumerate}

\item terme dispersif: $\square\phi +$ \boldmath $m^2 \phi $ \unboldmath  = 0 (Klein-Gordon)\\
\begin{enumerate}
\item onde plane: $k^2 \rightarrow k^2+m^2$

\end{enumerate}
\item terme non-linéaire: $\phi^3$ \\
\end{enumerate}
\end{block}
En s'éloignant de l'équation d'onde, les deux propriétés peuvent être conservées: ondes solitaires et solitons 

\end{frame}

%photo Russell pour le plaisir
\begin{frame}
Sur sa monture, John Russell poursuit sa destinée, vers l'onde solitaire!
\begin{columns}[T]
    \begin{column}[T]{.5\linewidth}
   \begin{figure}[0.3\textwidth]
 \includegraphics[scale=0.25]{russell.png}
    \end{figure}
    \end{column}
    \begin{column}[T]{.5\linewidth}
    \begin{figure}[0.3\textwidth]
   \includegraphics[scale=0.2]{soliton1b.png}
    \end{figure}
    \end{column}
  \end{columns}
\end{frame}

	\begin{frame}\frametitle{En image}

	\begin{figure}[0.3\textwidth]
   \includegraphics[scale=0.5]{densite.png}
    \end{figure}
    \end{frame}

\begin{frame}\frametitle{Description d'un soliton topologique}

	\begin{enumerate}
	\item Densité d'énergie d'un soliton: 
	\begin{enumerate}
	\item Localisée dans l'espace (finitude)
	\item Conservée et non-nulle (solution non-dissipative, exigence topologique)
	\end{enumerate}

				\begin{align*}
	E[\phi] = \int{dxdt (\mathcal{H}[\phi])} = \int{dxdt [\frac{1}{2} (\partial_x \phi)^2 + V(\phi)]}
	\end{align*}
	\item Énergie finie:
	\begin{enumerate}
	\item  $\lim\limits_{x \to \pm\infty}\partial_x \mathcal{H} =0$
	\item  $\lim\limits_{x \to \pm\infty} \phi[x] = g^{(i)}$ où les $g^{(i)}$ sont les min. de V	
	\end{enumerate}	
	\item Structure des vides non triviale
	\end{enumerate}

		\end{frame}

	




\subsection{Formalisme Lagrangien}
\begin{frame}

\frametitle{Formalisme Lagrangien}

\begin{block}{Notation covariante}

\begin{enumerate}
\item $x_0 = ct \hspace{1 cm} x_{1,2,3} = x,y,z$
\item $x_\mu = (x_0,\vec{x}) \hspace{1 cm} x^\mu = (x_0,-\vec{x})$
\item Métrique: $x^\mu = g^{\nu\mu} x_\nu$
\item Minkowski: $\eta^{\nu\mu} \rightarrow diag(1,-1,-1,-1)$
%\item indices répétés: $v_a \cdot v_a = \sum_{i=0}^{3}x_i^2$ (produit scalaire)
\end{enumerate}

\end{block}
\begin{align*}
\partial^\mu =& (\frac{1}{c} \partial_t, -\nabla) \\
\partial_\mu \partial^\mu =& \frac{1}{c^2} \partial_t^2- \nabla^2
\end{align*}
\end{frame}

\begin{frame}
\begin{enumerate}
\item \textit{Action}: $S[\phi] = \int{dt (L[\phi])}  =  \int{dx_\mu (\mathcal{L}[\phi])}$
\begin{enumerate}
\item Principe d'Hamilton: $\phi_0$ | action minimisée \\
\item Premier ordre nul pour un minimum d'action \\
\end{enumerate}
\item  \textit{Théorie des champs}:$\mathcal{L}[\phi] = \frac{1}{2} \partial_\mu \phi (\partial^\mu \phi)^* -V$
\item \textit{Euler-Lagrange:} $\partial_\mu \left(\frac{\mathcal{L}}{\partial(\partial_\mu\phi)}\right) = \frac{\partial\mathcal{L}}{\partial\phi}$
\end{enumerate}
\begin{exampleblock}{Équation d'ondes - V=0}
Par Euler-Lagrange:
\begin{align*}
 \partial_\mu(\frac{\partial_a\phi (\partial^a\phi)^* }{\partial_\mu \phi}) = 0 \\
\partial_\mu (\partial^\mu \phi)^*  = 0 \\
 \square \phi = 0 \\
\end{align*}

\end{exampleblock}

\end{frame}



\subsection{Kink}
\begin{frame}\frametitle{Kink: cas de figure typique}
\begin{columns}[T]
    \begin{column}[T]{.5\linewidth}
    \begin{exampleblock}{Sous: $\phi \rightarrow -\phi$ $(Z_2)$}
    \begin{align*}
    \mathcal{L} \rightarrow& \mathcal{L}\\
    \phi_0 \rightarrow& -\phi_0 \\
    \end{align*}    
    \end{exampleblock}
%    L'état de vide n'est pas invariant\\[0.5 cm]
    Théorie des champs:
\\$\rightarrow$ Champ scalaire $\phi$
    \\$\rightarrow$ 1+1 dimensions
  \\$\rightarrow$ solutions statiques \\[0.5 cm]
\end{column}
    \begin{column}[T]{.5\linewidth}
    \begin{block}{}
    \begin{align*}
      V(\phi) = \frac{\lambda}{4}(|\phi|^2 -\frac{m^2}{\lambda})^2
\end{align*}      
      \end{block}
    \begin{figure}
     \includegraphics[scale=0.4]{pot_z2.png}
    \end{figure}
    \end{column}
\end{columns}
\end{frame}

\begin{frame} \frametitle{Analogie mécanique classique}
\begin{equation*}
\mathcal{H}[\phi] = \frac{1}{2} \partial_\mu \phi (\partial_\mu \phi)^* +V
\end{equation*}
\begin{columns}[T]
    \begin{column}[T]{.55\linewidth}
    Champ
    \begin{enumerate}
    \item $\mathcal{H} =  \cancelto{}{\frac{1}{2}  (\partial_t \phi)^2} +  \frac{1}{2}  (\partial_x \phi)^2 +V(\phi) $
%    \item $\frac{\partial^2\phi}{\partial_x^2} = \frac{\partial V}{\partial\phi}$
    \item $E_\phi = \int{dx\left[\frac{1}{2}  (\partial_x \phi)^2 +V(\phi) \right]}$
    \end{enumerate}
    \end{column}
    \begin{column}[T]{.45\linewidth}
	Particule
    \begin{enumerate}
    \item $L =   \frac{1}{2}  \dot{q}^2 -U(q)$
%    \item $\frac{\partial^2\phi}{\partial_x^2} = -\frac{\partial U}{\partial\phi}$
    \item $S_q = \int{dt[\frac{1}{2}  \dot{q}^2 -U(q)] }$
    \end{enumerate}
    \end{column}
  \end{columns}
  \begin{figure}
\includegraphics[scale=0.5]{pot_inv.png}
\end{figure}
% 
%    $E_\phi$ finie $\leftrightarrow S_q$ finie $\rightarrow E_q = 0$\\
% On étudie alors: -V

  
\end{frame}

%\begin{frame}
%Récapitulons:
%\begin{enumerate}
%\item 1+1 dimensions, mais solution statique
%\item Potentiel V d'ordre 4 
%\item Équations d'Euler-Lagrange $\rightarrow$ Équations du mouvement: 
%\begin{enumerate}
%
%\item solution: kink
%\end{enumerate}
%\end{enumerate}
%\end{frame}

\begin{frame}\frametitle{Kink}
\begin{columns}[T]

 \begin{column}[T]{.6\linewidth}
 \begin{block}{}
  Par Euler-Lagrange: $\phi'' = \lambda \phi^3 - m^2 \phi$
\begin{align*}
\phi(x) = \frac{m}{\sqrt{\lambda}}tanh\left[\frac{m}{\sqrt{2}}(x-x_0)\right] \\
\end{align*}
 \end{block}
 
 \begin{exampleblock}{}
   
(de E-L): $\frac{1}{2} \phi'^2 = U(\phi)$  dans $\mathcal{H}(\phi)=\epsilon(x)$
\begin{align*}
\epsilon(x)= \frac{m^4}{\sqrt{2\lambda}}sech^4\left[\frac{m}{\sqrt{2}}(x-x_0)\right]
\end{align*}
 \end{exampleblock}

 \end{column}
 \begin{column}[T]{.4\linewidth}
\begin{figure}
\includegraphics[scale=0.3]{kink_energie.png}
\end{figure}
 \end{column}
\end{columns}
  \end{frame}

\begin{frame}

\begin{figure}
\includegraphics[scale=0.4]{soli_def.png}
\end{figure}

Charge topologique Q conservée (Q= $\int{k_0 dx}$)
\\$\rightarrow$ $Q\sim \phi(x=\infty)- \phi(x=-\infty)$
\\$\rightarrow$ Déformations continues impossibles (dont le temps)
\\$\rightarrow$ (Secteurs topologiques non connectés)\\[0.5 cm]

\textbf{Configuration non triviale de vides $\rightarrow$ Solution non dissipative}\\
Continuité de la solution en x:
\begin{align*}
\forall \hspace{1 mm} t \hspace{1 mm} \exists \hspace{1 mm} x \hspace{1 mm} | \hspace{1 mm} \epsilon(x,t) \geq U(\phi=0)\\
\rightarrow max|_x \epsilon(x,t) > 0
\end{align*}

%Mur de domaine (symétrie discrète brisée, 1 dimension) \\$\rightarrow$ Cordes cosmiques (cylindrique), monopôles (sphérique)

\end{frame}

\begin{frame}\frametitle{Retour - Cosmologie et solitons}
\begin{enumerate}
\item Symétries brisées dans la nature (\textbf{MS}, ferroaimant)
\end{enumerate}
\begin{enumerate}
\item Symétrie brisées: structure non triviale des vides dans l'univers\\
$\rightarrow$ défauts topologiques $\rightarrow$ solitons
\item Évolution de l'univers?
\end{enumerate}



\end{frame}

\begin{frame}\frametitle{Taux de désintégration du faux vide}
\begin{columns}[T]
    \begin{column}[T]{.45\linewidth}
        \begin{figure}[.45\linewidth]
    \includegraphics[scale=0.15]{false1.png}
    \end{figure}
    Effet tunnel quantique
    \\$\rightarrow$ Fluctuations quantiques $\leftrightarrow$ Fluctuations thermodynamiques (Bulles de vrai vide)
    \end{column}
    \begin{column}[T]{.45\linewidth}
       \begin{figure}
    \includegraphics[scale=0.15]{false2.png}
    \end{figure}
   Espace Euclidien:  Comportements quantiques dans Minkowski $\leftarrow$ Solutions classiques dans Euclide
 \\$\rightarrow$ min($S_E$) $\rightarrow$ Bounce, Instanton
 \\$\rightarrow$ Intégrale de chemin: $ \tau/V \approx e^{-So/\hbar}$    
    \end{column}
\end{columns}
\end{frame}


\begin{frame}
Fluctuation d'un soliton: autre source de désintégration!!!\\
\end{frame}



%-----------------------------------------------------------------------------------------------------------------

\section{Quotidien en cosmologie théorique des particules}
\begin{frame}
%potentiel avec lequel on travaille
\begin{block}{Potentiel à deux champs $\phi(x,t)$ et $\psi(x,t)$}
\begin{equation*}
V(\phi,\psi)=(\psi^2-\delta_1)(\psi^2-1)^2+\frac{\alpha}{\psi^2+\gamma}[(\phi^2-1)^2 - \frac{\delta_2}{4}(\phi-2)(\phi+1)^2] 
\end{equation*}

\end{block}
\begin{columns}[T]
    \begin{column}[T]{.5\linewidth}
  
\begin{enumerate}
\item 1+1 dimensions, on cherche une solution statique
\item Paramètres: $\alpha, \gamma, \delta_1, \delta_2$
\item $\gamma$: couplage
\item $\alpha$: Importance du 2ème terme 
\end{enumerate}  
    \end{column}
    \begin{column}[T]{.5\linewidth}
    \begin{figure}[0.3\textwidth]
    \includegraphics[scale=0.2]{Capture-2.png}
    \end{figure}
    \end{column}
  \end{columns}
\end{frame}

\begin{frame}
%Pourquoi bâtir un potentiel comme ça en premier lieu?!?!\\  

\begin{figure}[0.3\textwidth]
    \includegraphics[scale=0.25]{psi_phi.png}
    \end{figure}
\begin{columns}[T]
    \begin{column}[T]{.5\linewidth}
    \begin{enumerate}
    \item $\delta_1 \rightarrow$ contrôle du minimum central
    \item Ordre 6, CLASSIQUE! 
    \end{enumerate}
   
    \end{column}
    \begin{column}[T]{.5\linewidth}
    \begin{enumerate}
    \item $\delta_2 \rightarrow$ Contrôle de la séparation entre minimum sur l'axe $\phi$
    \end{enumerate}

    \end{column}
  \end{columns}

\end{frame}

\begin{frame}\frametitle{À venir...}
Solutions aux équations de mouvements (contraintes à $\lim\limits_{x \to \pm\infty}$)

\begin{columns}[T]
    \begin{column}[T]{.5\linewidth}
    \begin{figure}
    \includegraphics[scale=0.35]{traj.png}
    \end{figure}
    \end{column}

    \begin{column}[T]{.5\linewidth}
    \begin{figure}
    \includegraphics[scale=0.45]{phi_psi.png}
    \end{figure}
    \end{column}
\end{columns}
    
\end{frame}

\begin{frame}
Tester la stabilité de la solution

    \begin{columns}[T]
        \begin{column}[T]{.5\linewidth}
        \begin{figure}[0.3\textwidth]
     \includegraphics[scale=0.5]{bouncy.png}
    \end{figure}
        \end{column}
    
        \begin{column}[T]{.5\linewidth}
        \begin{figure}[0.3\textwidth]
     \includegraphics[scale=0.5]{ener.png}
    \end{figure}
        \end{column}
    \end{columns}

    
 Trouver une borne maximale sur l'action $\rightarrow$ borne minimale sur $\tau$
 
\end{frame}



\end{document}